\documentclass{article}
\usepackage[T1]{fontenc}
\usepackage[utf8]{inputenc}

% Linespread command allows you to change line spacing for the entire document
\linespread{1.18}

% Tweak page margins
\addtolength{\oddsidemargin}{-.875in}
\addtolength{\evensidemargin}{-.875in}
\addtolength{\textwidth}{1.75in}

\addtolength{\topmargin}{-.875in}
\addtolength{\textheight}{1.75in}

\usepackage{natbib}
\usepackage{hyperref}
\usepackage{xcolor}
\usepackage{xspace}
\usepackage{fancyhdr}
\hypersetup{
    colorlinks,
    linkcolor={red!50!black},
    citecolor={blue!50!black},
    urlcolor={blue!80!black}
}

\newcommand{\HRule}{\rule{\linewidth}{0.5mm}}
\newcommand{\Hrule}{\rule{\linewidth}{0.3mm}}

% Project specific macros
\newcommand{\graphite}{GRAPHITE\xspace}
\newcommand{\wave}{WAVE\xspace}

% School specific macros
\newcommand{\schoolShort}{Stanford\xspace}
\newcommand{\school}{Stanford\xspace}
\newcommand{\schoolLong}{Stanford University\xspace}

\newcommand{\profOne}{Prof. Matei Zaharia\xspace}
\newcommand{\profTwo}{Prof. Christos Kozyrakis\xspace}
\newcommand{\profThree}{Prof. Jure Leskovec\xspace}

% Creates header for each page
\usepackage{fancyhdr}
\pagestyle{fancy}
\fancyhf{}
\fancyhead[LE,RO]{\header\hskip\linepagesep\vfootline\thepage}
\newskip\linepagesep \linepagesep 5pt\relax
\def\vfootline{%
    \begingroup
    	\rule[-10pt]{0.75pt}{25pt}
    \endgroup
}
\def\header{%
	\begin{minipage}[]{120pt}
		\hfill Taeyang Kim 		% Applicant Name
    	\par \hfill 				% Formatting boilerplate
    	CS, MS, Fall 2025 			% Area, Program, Cycle, Year
    \end{minipage}
}
\fancyhead[RE,LO]{Statement of Purpose | \schoolLong}
\renewcommand\headrulewidth{0pt}

\begin{document}

%% Why do you wish to attend graduate school? What would you like to study? Keep it broad, details come-in later
A few years ago, I found myself in a challenging relationship with someone who struggled deeply with emotional turmoil. Despite her frequent bouts of distress, I approached each situation with patience, kindness, and a genuine desire to understand her pain. Over time, she confided in me about her past experiences with emotional manipulation and abuse, which had severely impacted her self-worth. This revelation was a pivotal moment for me. I became acutely aware of the profound impact that words and communication can have on an individual's psyche. This realization ignited my interest in \textbf{Natural Language Processing (NLP)} as a means to harness technology for emotional healing and support.
\\
% My decision to pursue graduate studies is a natural outcome of my experiences gained as a master's student at the \textbf{Indian Institute of Science (IISc)} and more recently as a research staff in the \textbf{Systems Group} at \textbf{Microsoft Research India}. I am interested in \textbf{distributed systems} and \textbf{data-intensive cloud computing} — in particular, the system-side problems associated with \textbf{learning and deploying machine learning models at scale}. My research endeavors have resulted in publications at well-known conferences such as \textbf{OSDI [1]} and \textbf{ICDE [3]}. Through graduate studies at \school, I wish to take the first step towards my long term goal of a career in research.

%% Describe 2-3 past projects that might be relevant to your research interests. (10-12 lines per project)

% PROJECT 1: P3 - Distributed Graph Neural Network Training at Scale
Determined to make a difference, I reached out to \textbf{Professor Nancy Fulda} at BYU's \textbf{Deep Representations and Architectures for Generative systems and Natural language understanding (DRAGN) Lab}. Under her guidance, I embarked on a project to develop a software application utilizing BYU's large language model, EVE. We focused on pretraining the model with curated content emphasizing self-worth, prevention strategies against physical and verbal abuse, and therapeutic counseling information. The goal was to create a tool that could provide users with resources for healing and empowerment. Our project's impact was recognized when we \textbf{won first place at the 2022 BYU ACM YHack Hackathon}.
\\
% In the recent past, there has been significant interest in Graph Neural Networks (GNNs)—neural networks that operate on graph structured data. Contemporary systems for \textbf{distributed GNN training} have been heavily influenced by, often building upon, existing systems for DNN training and graph processing systems. While in-principle this might look as the most natural path forward, retrofitting algorithms and leveraging tradeoffs designed in context of DNN training and/or graph processing systems lowers compute resource utilization due to communication stalls and more importantly lead to scalability bottlenecks.
% Based on this, I worked with \textbf{Anand Iyer} to build the $\mathbf{P^{3} ~[1]}$ system, which enables efficient distributed training of GNNs on large input graphs. $P^{3}$, published in \textbf{OSDI 2021}, effectively eliminates communication stalls by independently partitioning the graph along feature and structural dimension; unlike just the structural dimension, as done by most partitioners (e.g., METIS) designed for graph processing systems. Such independent partitioning scheme combined with intra-layer model parallelism and pipelining enables new push-pull based distributed training strategy that can outperform previous data parallel methods, achieving low communication and partitioning overhead, and high resource utilization. We are currently integrating pipelined push-pull parallelism in Microsoft's DeepGraph GNN training engine to train state-of-the-art models with thousands of GPUs.

% PROJECT 2: SURGEON - Early-Exit Inference
My passion for leveraging technology to address real-world problems extends beyond NLP. Witnessing my grandmother and other elderly farmers struggle with the physical demands of agriculture due to age and climate challenges, I was inspired to explore robotics and smart farming solutions. I contributed to the development of an autonomous farming robot capable of watering, fertilizing, and weed management through self-learning algorithms and computer vision. This innovation aimed to alleviate the burdens on aging farmers and enhance agricultural productivity in the face of global warming. Our efforts resulted in \textbf{winning the 2023 BYU ITCSA Raspberry Pi Competition}, an award traditionally secured by Electrical and Mechanical Engineering students.
\\
% More recently, I have been looking into \textbf{model serving systems}. Large-scale pre-trained language models such as BERT have brought significant improvements to NLP applications but at the cost of heavy computational burden. A growing body of work aims to alleviate this by exploiting the difference in the classification difficulty among samples and early-exiting at different stages of the network. 
% Although early-exit deep neural networks (EE-DNN) have demonstrated promising accuracy–latency trade-offs, we observe that the coarse-grained batching -- an essential technique to increase throughput, becomes suboptimal in effectively ensuring high hardware utilization as samples dynamically exit at different stages of the network. 
% Thus, with \textbf{Anand Iyer}, I have proposed \textbf{SURGEON [2]}, a system which takes EE-DNN model, throughput and latency requirements as inputs, and produces a model partition and service assignment across heterogeneous resources that satisfies SLA constraints using as few GPUs as possible. 
% Key idea in SURGEON is to consolidate two or more batches at partition boundary, thereby creating a larger batch which  improves hardware efficiency. SURGEON uses dynamic programming to adaptively identify when/how EE-DNN layers are to be partitioned and which/how-many devices are required to service each partition. I am currently evaluating the benefits of SURGEON for different EE-DNN model architectures and service-level objectives.

% PROJECT 3: Graphite - Distributed Temporal Graph Processing
At \textbf{Sandia National Laboratories}, I tackled cybersecurity challenges by integrating the \textbf{DistilBERT} model with \textbf{Transformer} architectures to classify security log data into 24 categories of cyberattacks. Rapid detection and classification are crucial for organizations to mitigate threats and prevent financial losses. Our model achieved an impressive 99.7\% accuracy and was honored with \textbf{first place in the Machine Learning category at the 2024 BYU Capstone Celebration Competition.}
\\

During my tenure at \textbf{Pattern Inc.}, a unicorn company specializing in e-commerce solutions, I noticed that support engineers were overwhelmed with repetitive IT tickets. To streamline this process, I pre-trained \textbf{GPT-3.5} and \textbf{LLaMA2} models but found they lacked precision in generating accurate responses. Through research, I implemented a \textbf{Retrieval-Augmented Generation (RAG)} approach combined with \textbf{word embedding} techniques. By vectorizing a bespoke dataset of questions and answers, we automated the IT ticket resolution system, reducing resolution times by 87\% and workload by 78\%. This significant improvement earned me the \textbf{"Employee of the Year"} award among 1780 employees at the Accelerate 24 event, one of the best ECommerce Conferences.
\\

My entrepreneurial spirit led me to co-found \textbf{Glöd AI}, the world's first AI-driven product advertisement video generation platform. Leading a team of ten engineers and product managers, we harnessed the power of machine learning to revolutionize marketing content creation. Additionally, as the team leader of the Tech Team in BYU \textbf{Korean Business Students Association (KBSA)}, I guided our team to \textbf{victory at the 2024 BYU ACM Hackathon} with our AI Shorts Generation Platform.
\\

Service to others has always been a cornerstone of my journey. Serving as an Operations Specialist in the Korean Augmentation To the United States Army (KATUSA), I was honored with the \textbf{General Paik Sun Yup Leadership Award}, \textbf{two US Army Commendation Medals}, and the \textbf{Best KATUSA Award}. My military service taught me leadership, discipline, and the importance of collaboration. Furthermore, I dedicated time as a full-time volunteer in Melbourne, Australia, self-funding my efforts to support individuals battling addictions to drugs, alcohol, and other detrimental behaviors.
\\

\textbf{Research Interests}\\
My experiences have solidified my research interests at the intersection of \textbf{Natural Language Processing} and \textbf{Robotics}, specifically focusing on \textbf{Human-Robot Interaction}  to address pressing societal issues. One area of concern is addiction recovery; approximately 50\% of Americans aged 12 and older have used illicit drugs at least once, and 14\% struggle with pornography addiction. Overcoming these addictions often requires costly psychological and medical interventions that may not be accessible or appealing to everyone. I envision integrating NLP with robotics to create empathetic companions for individuals battling addiction. By leveraging advanced NLP techniques to recognize and respond to emotional states, these robots could offer nonjudgmental support and companionship akin to a supportive family member or pet, thereby reducing barriers to seeking help and democratizing access to care.
\\

Additionally, I aim to develop machine learning models that predict crop growth based on evolving weather patterns and climate data. By integrating these predictive models with robotics and smart farming technologies, we can optimize agricultural processes like planting, irrigation, and harvesting. Creating autonomous farming robots capable of tasks such as watering, fertilizing, and weed management would enhance crop yields and reduce the physical burden on farmers—particularly benefiting older farmers like my grandmother who struggle with the impacts of climate change. Collaborating with experts in various fields, I hope to develop scalable solutions that address both environmental challenges and global food scarcity.
\\

\textbf{Conclusion} \\
Stanford University's Master of Science in Computer Science program is the ideal environment for me to pursue these ambitions. I am particularly drawn to \textbf{Professor Diyi Yang}'s work as the Director of the \textbf{Social and Language Technologies Lab}. Her research on developing NLP solutions to address societal challenges resonates deeply with my goal of creating empathetic AI counseling tools. Additionally, I am eager to work with \textbf{Professor Jeannette Bohg}, who directs the \textbf{Interactive Perception and Robot Learning Lab}. Her exploration of robust sensorimotor coordination in humans and its implementation on robots, particularly in robotic grasping and manipulation, aligns with my interest in developing therapeutic robotic companions that interact seamlessly with users. Furthermore, \textbf{Professor Dorsa Sadigh}'s research at the intersection of robotics and machine learning, focusing on interactive robot learning and building robots that learn from and adapt to humans, complements my goal of creating adaptive and human-aligned robotic systems. Collaborating within these esteemed labs and \textbf{Stanford Robotics Center} that was built this year offers the opportunity to meld cutting-edge NLP and robotics research to create impactful solutions that enhance human well-being.
\\

Looking beyond the master's program, I aspire to lead initiatives that bridge government, academia, and industry to combat pervasive issues such as addiction, trauma from abuse, and mental health challenges. By harnessing technology, knowledge, and compassionate leadership, I aim to establish an organization dedicated to providing accessible support systems worldwide. Stanford's resources, network, and culture of innovation will be instrumental in achieving this vision.
\\

My unwavering commitment to helping others is the driving force behind my application to Stanford. I believe that with the advanced education and collaborative opportunities the program offers, I can make significant strides in developing technologies that bring healing and support to those in need. No obstacle is insurmountable when one's purpose is rooted in genuine care for others, and I am eager to contribute to the Stanford community while advancing this mission.

% Before MSR, I completed my master's (by research) in Computer and Data Systems at the \textbf{Indian Institute of Science (IISc), Bangalore} advised by \textbf{Yogesh Simmhan}. At IISc, my thesis focused on system-side optimizations for \textbf{distributed temporal\footnote{Graphs whose structure and attributes evolve over time} graph analytics}.
% Despite their growing availability, existing abstractions and frameworks do not scale well due to redundant computing and/or messaging across time-points.
% We address this gap through \textbf{\graphite[3]}, which uses time-interval as the data-parallel unit of computation.
% \graphite relies on our novel time-warp operator, which automatically partitions a vertex’s temporal state, and temporally aligns and groups messages to these states. This eases the temporal reasoning required by the user logic, and avoids redundant execution of user logic and messaging within an interval to provide key performance benefits.
% This work was published in \textbf{ICDE 2020} and is currently used at IISc, IIT-Jodhpur, and IIIT-Hyderabad for contact tracing and analytics over temporal graphs\footnote{\url{https://covid19.iisc.ac.in/gocoronago-contact-tracing-app-and-network-analytics/}}.
% \textbf{\wave[4]}, an extension of \graphite, incorporates \textbf{dependency-driven incremental processing} by tracking dependencies to capture how intermediate values get computed, and then uses this information to incrementally propagate the impact of change across intermediate values.
% \wave was presented at the \textbf{2nd ACM Student Research Competition (Graduate Category)} co-located with SOSP 2019. Out of the 15 participating submissions, \wave was one of the 5 finalists and subsequently awarded the Bronze Medal.

%% Non-research accomplishments (e.g. Grades, Academic Service, Work experience) (10-12 lines)

% % Grades
% I realize the need for a strong theoretical foundation to pursue advanced research. In this direction, I have always striven for academic excellence - I stood top of the class during both bachelor's and master's studies.
% % TA and Academic Service
% My time at IISc offered me opportunities to assist with two graduate courses, and participate in Artifact Evaluation Committee (AEC)\footnote{AEC SOSP 2019, OSDI 2020 and ASPLOS 2020} and Shadow Program Committee\footnote{Shadow PC EuroSys 2021 and 2022} at several conferences. These experiences have taught me how to organize and articulate ideas more effectively, participating in faux-PC discussion showed me how I can better interpret the subtext of reviews and comments before/after rebuttal, and seeing the difference between the submitted papers I reviewed and the accepted papers presented at EuroSys 2021 gave me insight into what the authors may have learned from reviewer feedback.
% % Industry
% The time I spent in the industry prior to and after attending graduate school helped me earn valuable soft skills - time management, team work and co-operation, which I believe are vital for survival in tough academic settings.

%% Why this school? List professors you would like to work with and why? (10-12 Lines)

% I believe my experience with data-intensive systems at Microsoft Research has offered me a unique perspective into practical problems experienced by developers and cloud operators in deploying, operating, and monitoring computer systems at scale, and makes me uniquely qualified to tackle the big picture questions in this area.
% At \schoolShort, I would like to continue to work on efficient \textbf{systems infrastructure and tooling for emerging data-intensive workloads}.
% As machine learning models become larger and are increasingly used in safety and performance critical applications, demand for compute grows, and hardware accelerators evolve rapidly, I see this as an exciting area to work on from a number of angles such as performance, resiliency, resource-efficiency, and/or affordability.
% \schoolShort's demonstrated leadership in data-intensive systems research, its exemplary faculty, as well as unique inter-disciplinary atmosphere make it the ideal environment for me to conduct my graduate study.. 
% \textbf{\profOne}, \textbf{\profTwo}, and \textbf{\profThree} are faculty I would especially like to work with. Over the years, I have been influenced by several of their research works on new approaches and programming models for cloud-computing, frameworks for efficient deep learning training, and real-time model serving (e.g., Spark, Snorkel, Shinjuku, PipeDream, ROC, INFaaS, GNNAutoScale, GraphSAGE), and some of the insights I have been able to draw from these works have become a natural part of my master's thesis and research papers. An opportunity to work with them, is extremely appealing to me as a budding systems researcher.

%% Summary (3-4 Lines)

% In summary, I believe I bring with me research experience, industry-sharpened programming and soft skills, and above all, an insatiable desire to learn and excel. I look forward to the next milestone in my life -- a PhD in Computer Science from \school. 

% Add some blank space between text and references
\vspace{0.125in}

% References

% **NOTE**: There are better ways to manage citations in LaTeX, most notably using a bibTeX. I wanted to have greater control on how citations were spaced and formatted and therefore ended up hardcoding them here. Your mileage may wary!

% [1] \underline{Swapnil Gandhi}, Anand Padmanabha Iyer, ``P3: Distributed Deep Graph Learning at Scale'', In proceedings of the 15th USENIX Symposium on Operating Systems Design and Implementation
% (\textbf{OSDI 2021}), July 2021. \url{https://swapnilgandhi.com/papers/p3-osdi21.pdf}

% [2] \underline{Swapnil Gandhi}, Anand Padmanabha Iyer, ``SURGEON: Fast \& Efficient DNN Inference Using Practical Early-Exit Networks'' (\textbf{On-going Project})

% [3] \underline{Swapnil Gandhi}, Yogesh Simmhan, ``An Interval-centric Model for Distributed Computing over Temporal Graphs'', In proceedings of the 36th IEEE International Conference on Data Engineering
% (\textbf{ICDE 2020}), Dallas, Texas, April 2020. \url{https://swapnilgandhi.com/papers/icm-icde20.pdf}

% [4] \underline{Swapnil Gandhi}, ``Wave: A Substrate for Distributed Incremental Graph Processing on Commodity Clusters'', 2nd ACM Student Research Competition (\textbf{SRC}) at 27th Symposium on Operating Systems Principles (\textbf{SOSP 2019}), Ontario, Canada, Oct 2019. \url{https://swapnilgandhi.com/papers/wave-sosp19.pdf}

\end{document}

% That's All Folks.

% Best of luck, you got this! :)
