\documentclass{article}

\usepackage[T1]{fontenc}
\usepackage[utf8]{inputenc}

% Linespread command allows you to change line spacing for the entire document
\linespread{1.18}

% Tweak page margins
\addtolength{\oddsidemargin}{-.875in}
\addtolength{\evensidemargin}{-.875in}
\addtolength{\textwidth}{1.75in}

\addtolength{\topmargin}{-.875in}
\addtolength{\textheight}{1.75in}

\usepackage[numbers,super]{natbib}
\usepackage{hyperref}
\usepackage{xcolor}
\usepackage{xspace}
\usepackage{fancyhdr}
\hypersetup{
    colorlinks,
    linkcolor={red!50!black},
    citecolor={blue!50!black},
    urlcolor={blue!80!black}
}

\makeatletter
\def\NAT@biblabelnum#1{[#1]}
\def\NAT@cite@super#1{\textsuperscript{[#1]}}
\makeatother

\newcommand{\HRule}{\rule{\linewidth}{0.5mm}}
\newcommand{\Hrule}{\rule{\linewidth}{0.3mm}}

% Project specific macros
\newcommand{\graphite}{GRAPHITE\xspace}
\newcommand{\wave}{WAVE\xspace}

% School specific macros
\newcommand{\schoolShort}{Stanford\xspace}
\newcommand{\school}{Stanford\xspace}
\newcommand{\schoolLong}{Stanford University\xspace}

\newcommand{\profOne}{Prof. Matei Zaharia\xspace}
\newcommand{\profTwo}{Prof. Christos Kozyrakis\xspace}
\newcommand{\profThree}{Prof. Jure Leskovec\xspace}

\newcommand{\portfolioWebsite}{\href{taeyangkim.info}{taeyangkim.info}}


% Creates header for each page
\usepackage{fancyhdr}
\pagestyle{fancy}
\fancyhf{}
\fancyhead[LE,RO]{\header\hskip\linepagesep\vfootline\thepage}
\newskip\linepagesep \linepagesep 5pt\relax
\def\vfootline{%
    \begingroup
    	\rule[-10pt]{0.75pt}{25pt}
    \endgroup
}
\def\header{%
	\begin{minipage}[]{120pt}
		\hfill Taeyang Kim 		% Applicant Name
    	\par \hfill 				% Formatting boilerplate
    	CS, MS, Fall 2025 			% Area, Program, Cycle, Year
    \end{minipage}
}
\fancyhead[RE,LO]{%
    \begin{tabular}{@{}l@{}}
        Statement of Purpose | \schoolLong \\
        \portfolioWebsite
    \end{tabular}
}
\renewcommand\headrulewidth{0pt}
\setlength{\parindent}{0pt}

\begin{document}

%% Why do you wish to attend graduate school? What would you like to study? Keep it broad, details come-in later
A few years ago, I found myself in a challenging relationship with someone who struggled deeply with emotional turmoil. Despite her frequent bouts of distress, I approached each situation with patience, kindness, and a genuine desire to understand her pain. Over time, she confided in me about her past experiences with emotional manipulation and abuse, which had severely impacted her self-worth. This revelation was a pivotal moment for me. I became acutely aware of the profound impact that words and communication can have on an individual's psyche. This realization ignited my interest in leveraging technology, particularly \textbf{Natural Language Processing (NLP)}, to address critical societal issues and empower individuals.
\\

\Large\textbf{1\hspace{1em}Research Experience}\normalsize
\\
\\
% BYU DRAGN Lab
\textbf{Abuse Prevention and Emotional Support via NLP.} Motivated by a commitment to support individuals facing abuse, I collaborated with Professor Nancy Fulda at BYU DRAGN Lab. Under her guidance, I led a project to develop a software application utilizing BYU's large language model, EVE. We pretrained EVE with curated content centered on self-worth, abuse prevention strategies, and therapeutic counseling. Our aim was to create an accessible tool that could offer personalized emotional support and resources for healing and empowerment. This endeavor not only deepened my understanding of language models and their societal applications but also reinforced my dedication to harnessing NLP for social good. Our project's impact was recognized with \textbf{first place at the 2022 BYU ACM YHack Hackathon}\cite{byuacm2022}.
\\

% Sandia National Laboratories
\textbf{Cybersecurity Attack Classification via NLP.} Building upon my passion for applying NLP to critical issues and safeguarding communities, I joined Sandia National Laboratories to tackle cybersecurity challenges. Recognizing the escalating threats organizations face, I focused on rapid detection and classification of cyberattacks using NLP techniques. I integrated the DistilBERT model with Transformer architectures to classify security log data into 24 categories of cyberattacks. This fusion of NLP and cybersecurity enabled more efficient threat identification, crucial for mitigating risks and preventing financial losses. Our model achieved a 99.7\% accuracy rate and was honored with \textbf{first place in the Machine Learning category at the 2024 BYU Capstone Celebration Competition}\cite{cybersecurity}. This experience solidified my interest in interdisciplinary applications of NLP and demonstrated its versatility in enhancing security measures.
\\

% Pattern Inc. Experience
\textbf{Automating IT Ticket Resolution via NLP.} While working at Pattern Inc., an E-Commerce solutions company, I noticed support engineers overwhelmed by repetitive IT tickets—a challenge that hindered efficiency and customer satisfaction. Drawing on my NLP expertise, I sought to streamline this process. Initial attempts with GPT-3.5 and LLaMA2 models revealed limitations in precision for generating accurate responses. Through research, I implemented a Retrieval-Augmented Generation (RAG) approach combined with word embedding techniques. By vectorizing a custom dataset of frequently asked questions and answers, we automated the IT ticket resolution system. This innovation reduced resolution times by 87\% and decreased workload by 78\%, significantly improving operational efficiency. Our success was recognized when I received the \textbf{"Employee of the Year"} award among 1,780 employees at the Accelerate 24 event\cite{pattern}. This project reinforced my belief in NLP's capability to optimize processes and enhance user experiences.
\\

% Smart Farming Robotics
\textbf{Smart Farming Robotics with Computer Vision.} Motivated by personal experiences with childhood food insecurity, I sought to address global hunger through technological innovation. I engineered a smart farming robot that autonomously and precisely distributes water, fertilizer, and weed killer, utilizing Raspberry Pi, Arduino, LiDAR sensors, linear actuators, and stepper motors. Although this project diverges from my primary focus on NLP, it reflects my broader commitment to leveraging technology for societal good. Furthermore, it sparked my interest in integrating NLP with robotics to create more intuitive human-robot interactions. Our innovation \textbf{won the 2023 BYU ITCSA Raspberry Pi Competition}, highlighting the impact of interdisciplinary approaches\cite{raspberrypi}.


\Large\textbf{2\hspace{1em}Research Interests}\normalsize
\\
\\
These diverse experiences have equipped me with a unique perspective and a deep understanding of the challenges faced by individuals from various backgrounds. I am eager to leverage these insights to develop innovative solutions on \textbf{Human-Robot Interaction} integrating \textbf{Natural Language Processing} and \textbf{Robotics}. 
\\

\textbf{Addiction Recovery Counseling Robot.} Approximately 24.9\% of Americans aged 12 and older have used illicit drugs last year, and 14\% struggle with pornography addiction\cite{health}. To enhance the robot's ability to provide empathetic support, I propose integrating computer vision techniques to analyze users' facial expressions and body language alongside their verbal communication with NLP. By employing advanced neural networks for emotion recognition, the robot can detect subtle cues of discomfort, anxiety, or distress that may not be evident through words alone. This multimodal analysis enables the creation of a more responsive and understanding companion, capable of adapting its interactions to the user's emotional needs in real-time. This innovation not only makes counseling more effective but also provides a unique approach to supporting individuals battling addiction, aligning with my goal of leveraging technology to make mental health care more accessible and personalized.
\\

\textbf{Depression Therapeutic Robot.} As 280 million people in the world suffer from depression, I aim to develop a therapeutic robot that provides warm, gentle physical comfort through motions like hugging and patting, simulating human touch to improve emotional well-being for individuals with depression \cite{depression}. By integrating thermal elements, LiDAR sensor, tactile sensors, and haptic feedback, the robot can mimic human gestures, while reinforcement learning enables it to adapt its interactions based on the user's verbal and non-verbal responses. Utilizing real-time emotional analysis with computer vision and NLP, the robot adjusts its warmth and touch according to the user's emotional state. This personalized, evolving support system complements traditional therapy, making mental health support more accessible and improving the quality of life for those struggling with depression.
\\

\Large\textbf{3\hspace{1em}Conclusion}\normalsize
\\
\\
Stanford University's Master of Science in Computer Science program is the ideal environment for me to pursue these ambitions. I am particularly drawn to \textbf{Professor Diyi Yang}'s work as the Director of the \textbf{Social and Language Technologies Lab}. Her research on developing NLP solutions to address societal challenges resonates deeply with my goal of creating empathetic AI counseling tools\cite{diyiyang}. Additionally, I am eager to work with \textbf{Professor Dorsa Sadigh} on her research at the intersection of robotics and machine learning in \textbf{Stanford Intelligent and Interactive Autonomous Systems Group (ILIAD)}, focusing on interactive robot learning and building robots that learn from and adapt to humans, complements my goal of creating adaptive and human-aligned robotic systems\cite{grannen2024vocal}. Furthermore, \textbf{Professor Jeannette Bohg}, who directs the \textbf{Interactive Perception and Robot Learning Lab}, explores on robust sensorimotor coordination in humans and its implementation on robots, particularly in robotic grasping and manipulation, grabbed my attention as it aligns with my interest in developing therapeutic robotic companions that interact seamlessly with users\cite{srinivasan2024dexmots}.
\\

Looking beyond the master's program, I aspire to lead initiatives that bridge government, academia, and industry to combat pervasive issues such as addiction, trauma from abuse, and mental health challenges. By harnessing technology, knowledge, and compassionate leadership, I aim to establish an organization dedicated to providing accessible support systems worldwide. Stanford's resources, network, and culture of innovation will be instrumental in achieving this vision.

\bibliographystyle{unsrtnat}
\bibliography{references}

\end{document}
